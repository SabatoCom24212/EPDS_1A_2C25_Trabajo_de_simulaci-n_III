% Options for packages loaded elsewhere
% Options for packages loaded elsewhere
\PassOptionsToPackage{unicode}{hyperref}
\PassOptionsToPackage{hyphens}{url}
\PassOptionsToPackage{dvipsnames,svgnames,x11names}{xcolor}
%
\documentclass[
  english,
  11pt,
  a4paper,
]{article}
\usepackage{xcolor}
\usepackage[top=2.5cm,bottom=2.5cm,left=2.5cm,right=2.5cm]{geometry}
\usepackage{amsmath,amssymb}
\setcounter{secnumdepth}{3}
\usepackage{iftex}
\ifPDFTeX
  \usepackage[T1]{fontenc}
  \usepackage[utf8]{inputenc}
  \usepackage{textcomp} % provide euro and other symbols
\else % if luatex or xetex
  \usepackage{unicode-math} % this also loads fontspec
  \defaultfontfeatures{Scale=MatchLowercase}
  \defaultfontfeatures[\rmfamily]{Ligatures=TeX,Scale=1}
\fi
\usepackage{lmodern}
\ifPDFTeX\else
  % xetex/luatex font selection
\fi
% Use upquote if available, for straight quotes in verbatim environments
\IfFileExists{upquote.sty}{\usepackage{upquote}}{}
\IfFileExists{microtype.sty}{% use microtype if available
  \usepackage[]{microtype}
  \UseMicrotypeSet[protrusion]{basicmath} % disable protrusion for tt fonts
}{}
\makeatletter
\@ifundefined{KOMAClassName}{% if non-KOMA class
  \IfFileExists{parskip.sty}{%
    \usepackage{parskip}
  }{% else
    \setlength{\parindent}{0pt}
    \setlength{\parskip}{6pt plus 2pt minus 1pt}}
}{% if KOMA class
  \KOMAoptions{parskip=half}}
\makeatother
% Make \paragraph and \subparagraph free-standing
\makeatletter
\ifx\paragraph\undefined\else
  \let\oldparagraph\paragraph
  \renewcommand{\paragraph}{
    \@ifstar
      \xxxParagraphStar
      \xxxParagraphNoStar
  }
  \newcommand{\xxxParagraphStar}[1]{\oldparagraph*{#1}\mbox{}}
  \newcommand{\xxxParagraphNoStar}[1]{\oldparagraph{#1}\mbox{}}
\fi
\ifx\subparagraph\undefined\else
  \let\oldsubparagraph\subparagraph
  \renewcommand{\subparagraph}{
    \@ifstar
      \xxxSubParagraphStar
      \xxxSubParagraphNoStar
  }
  \newcommand{\xxxSubParagraphStar}[1]{\oldsubparagraph*{#1}\mbox{}}
  \newcommand{\xxxSubParagraphNoStar}[1]{\oldsubparagraph{#1}\mbox{}}
\fi
\makeatother


\usepackage{longtable,booktabs,array}
\usepackage{calc} % for calculating minipage widths
% Correct order of tables after \paragraph or \subparagraph
\usepackage{etoolbox}
\makeatletter
\patchcmd\longtable{\par}{\if@noskipsec\mbox{}\fi\par}{}{}
\makeatother
% Allow footnotes in longtable head/foot
\IfFileExists{footnotehyper.sty}{\usepackage{footnotehyper}}{\usepackage{footnote}}
\makesavenoteenv{longtable}
\usepackage{graphicx}
\makeatletter
\newsavebox\pandoc@box
\newcommand*\pandocbounded[1]{% scales image to fit in text height/width
  \sbox\pandoc@box{#1}%
  \Gscale@div\@tempa{\textheight}{\dimexpr\ht\pandoc@box+\dp\pandoc@box\relax}%
  \Gscale@div\@tempb{\linewidth}{\wd\pandoc@box}%
  \ifdim\@tempb\p@<\@tempa\p@\let\@tempa\@tempb\fi% select the smaller of both
  \ifdim\@tempa\p@<\p@\scalebox{\@tempa}{\usebox\pandoc@box}%
  \else\usebox{\pandoc@box}%
  \fi%
}
% Set default figure placement to htbp
\def\fps@figure{htbp}
\makeatother



\ifLuaTeX
\usepackage[bidi=basic]{babel}
\else
\usepackage[bidi=default]{babel}
\fi
% get rid of language-specific shorthands (see #6817):
\let\LanguageShortHands\languageshorthands
\def\languageshorthands#1{}
\ifLuaTeX
  \usepackage[english]{selnolig} % disable illegal ligatures
\fi


\setlength{\emergencystretch}{3em} % prevent overfull lines

\providecommand{\tightlist}{%
  \setlength{\itemsep}{0pt}\setlength{\parskip}{0pt}}



 


\usepackage[utf8]{inputenc}
\usepackage[T1]{fontenc}
\usepackage{fancyhdr}
\setlength{\headheight}{14pt}
\addtolength{\topmargin}{-2pt}
\pagestyle{fancy}
\fancyhead[L]{Trabajo de Simulación III}
\fancyhead[R]{IFTS N° 29}
\usepackage{float}
\floatplacement{figure}{H}
\usepackage{graphicx}
\makeatletter
\@ifpackageloaded{caption}{}{\usepackage{caption}}
\AtBeginDocument{%
\ifdefined\contentsname
  \renewcommand*\contentsname{Table of contents}
\else
  \newcommand\contentsname{Table of contents}
\fi
\ifdefined\listfigurename
  \renewcommand*\listfigurename{List of Figures}
\else
  \newcommand\listfigurename{List of Figures}
\fi
\ifdefined\listtablename
  \renewcommand*\listtablename{List of Tables}
\else
  \newcommand\listtablename{List of Tables}
\fi
\ifdefined\figurename
  \renewcommand*\figurename{Figure}
\else
  \newcommand\figurename{Figure}
\fi
\ifdefined\tablename
  \renewcommand*\tablename{Table}
\else
  \newcommand\tablename{Table}
\fi
}
\@ifpackageloaded{float}{}{\usepackage{float}}
\floatstyle{ruled}
\@ifundefined{c@chapter}{\newfloat{codelisting}{h}{lop}}{\newfloat{codelisting}{h}{lop}[chapter]}
\floatname{codelisting}{Listing}
\newcommand*\listoflistings{\listof{codelisting}{List of Listings}}
\makeatother
\makeatletter
\makeatother
\makeatletter
\@ifpackageloaded{caption}{}{\usepackage{caption}}
\@ifpackageloaded{subcaption}{}{\usepackage{subcaption}}
\makeatother
\makeatletter
\@ifpackageloaded{tcolorbox}{}{\usepackage[skins,breakable]{tcolorbox}}
\makeatother
\makeatletter
\@ifundefined{shadecolor}{\definecolor{shadecolor}{HTML}{31BAE9}}{}
\makeatother
\makeatletter
\makeatother
\makeatletter
\ifdefined\Shaded\renewenvironment{Shaded}{\begin{tcolorbox}[sharp corners, borderline west={3pt}{0pt}{shadecolor}, enhanced, interior hidden, frame hidden, boxrule=0pt, breakable]}{\end{tcolorbox}}\fi
\makeatother
\usepackage{bookmark}
\IfFileExists{xurl.sty}{\usepackage{xurl}}{} % add URL line breaks if available
\urlstyle{same}
\hypersetup{
  pdflang={en},
  colorlinks=true,
  linkcolor={black},
  filecolor={Maroon},
  citecolor={blue},
  urlcolor={blue},
  pdfcreator={LaTeX via pandoc}}


\author{}
\date{}
\begin{document}

\begin{titlepage}
\centering
% LOGO DE LA INSTITUCIÓN
\includegraphics[width=0.3\textwidth]{logo.png}\\[2cm]

% TÍTULO DEL TRABAJO
{\Huge\bfseries Trabajo de Simulación III\par}
\vspace{1cm}

% INSTITUCIÓN
{\Large Instituto de Formación Técnica Superior N° 29\par}
\vspace{0.5cm}

% ASIGNATURA
{\large Estadística y Probabilidad para el Desarrollo de Software\par}
\vspace{2cm}

% AUTORES
{\large
Rinaldi, Flavio\\
Sabato, Ángel\\
Shifman, Iván\\
Zárate, Marcelo\\
\par}

\vfill

\end{titlepage}

\renewcommand*\contentsname{Índice de Contenidos}
{
\hypersetup{linkcolor=}
\setcounter{tocdepth}{3}
\tableofcontents
}

\section{Trabajo de simulación 3}\label{trabajo-de-simulaciuxf3n-3}

\subsection{Consigna}\label{consigna}

Para el último trabajo de simulación, vamos a hacer un análisis de datos
a partir de la encuesta de salarios de Sysarmy del primer trimestre de
2025 (si están publicados los del segundo trimestre o posterior, podemos
usar esos, pero a agosto de 2025 todavía no estaban).

Los datos oficiales se encuentran
\href{https://docs.google.com/spreadsheets/d/1hlLwv9SLJvrnsTq_UsEAHkHGNiziH7IdT1lJd4fq6kU/edit?gid=1462536742\#gid=1462536742}{acá}.
Por si en algún momento el enlace se cae o cambia de ubicación, una
versión alojada en mi drive puede encontrarse
\href{https://docs.google.com/spreadsheets/d/1Vq9F6xE03fR0x6pXlRMDN_bww7NWItjMW0U89qV3AvM/edit?usp=sharing}{acá}.

Las consignas de este trabajo no son tan dirigidas como las de los
trabajos anteriores, pues en el análisis de datos, siempre hay libertad
y margen para la creatividad y la producción personal. Sin embargo, les
compartimos algunas pautas de lo que debe tener, como mínimo, este
trabajo.

Pautas generales y \textbf{OBLIGATORIAS} para la aprobación de la
entrega:

\begin{itemize}
\item
  Debe replicarse, como mínimo, un análisis similar al aquí presentado,
  para estos datos (SysArmy 2025 o cualquier otro dataset que sea de su
  interés). Por ``replicar'' nos referimos a que el análisis debe
  incluir: inspección y limpieza de los datos, descripción y
  visualización, o estudio de alguna variable de interés a partir de
  alguna hipótesis o conjetura.
\item
  Debe escribirse en formato ``informe'', es decir, no se trata de
  exhibir código y gráficos, sino de explicar qué se observa y por qué
  es relevante observar eso. El informe es requisito
  \textbf{excluyente}. No se aprueba el trabajo de simulación sin él.
  Este informe breve debe entregarse en pdf, en esta entrega \textbf{NO}
  se evalúa el \emph{colab}, sino el reporte. No tiene que ser largo, al
  contrario, tiene que ser de calidad. Como dice el dicho: lo bueno -si
  breve- dos veces bueno\ldots{}
\item
  El trabajo debe contener, como mínimo, \textbf{una conjetura que sea
  sometida a prueba y de la que se exhiba alguna conclusión
  fundamentada}, como se hizo en el caso de los datos de 2020 para el
  salario medio bruto por género y para hombres y mujeres con nivel
  universitario completo. Por ejemplo, frente a la pregunta de si el
  salario medio de mujeres y hombres es igual, podríamos poner en
  práctica lo que estudiamos sobre convergencia para, de alguna forma,
  darnos una idea de cuán probable es observar lo que efectivamente
  estamos observando. Este es un ``coqueteo'' con la estadística
  inferencial, que no estudiamos formalmente en la materia, pero que es
  válido comenzar a encarar con todo lo que hemos estudiado. Esta
  conjetura puede hacerse con datos propios, si es que eligen trabajar
  con otro dataset.
\end{itemize}

El resto de la producción queda a criterio de los grupos. Esperamos que
haya un interés genuino en tratar de extraer información a partir de
estos datos. ¡Muchos éxitos!

\textbf{PD: El formato ``informe'' puede ser cambiado por el formato
``póster/infografía'' si es que así prefieren.}

\subsection{\texorpdfstring{Nuestra resolución: \emph{Optimización
salarial}}{Nuestra resolución: Optimización salarial}}\label{nuestra-resoluciuxf3n-optimizaciuxf3n-salarial}

Para realizar nuestro análisis tomamos la perspectiva de un profesional
del sector IT que necesita transfomar su carrera por alguna razón.
Sabemos que esta industria se caracteriza por un nivel de rotación alto
y los recorridos profesionales suelen ser muy variados. La pregunta
central de nuestro trabajo es \emph{¿cómo puedo maximizar mi salario?},
es decir, \emph{¿cuál es la \textbf{combinación ganadora} que se podría
obtener como insight de este dataset?} Los resultados obtenidos podrían
ser de utilidad tanto para el \emph{junior}, que necesita alguna
``brújula'' para ver por dónde orientarse dentro del mercado laboral,
como para el \emph{senior} que quizá se siente estancado y necesite
reorientar su desarrollo profesional dentro del sector.

\subsubsection{Inspección y limpieza de
datos}\label{inspecciuxf3n-y-limpieza-de-datos}

Comenzamos importando las librerías necesarias y cargando el dataset.
Luego inspeccionamos los datos para ver qué variables tenemos
disponibles y qué tipo de datos contienen.

\begin{verbatim}
*** Archivo cargado exitosamente
    Dimensiones: 5196 filas x 50 columnas
\end{verbatim}

Luego procesamos los datos para dejarlos en un formato adecuado para el
análisis. Esto incluye manejar valores faltantes, convertir variables
categóricas en numéricas (si es necesario) y asegurarnos de que todas
las variables estén en el tipo de dato correcto.

\begin{verbatim}
[PASO 1] Calculando salarios en USD con TC individual...

    * 927 registros con TC reportado individual
    * TC mediano de la encuesta: $1,060.00
    * 4269 registros usando TC mediano

Dataset preparado: (5196, 53)
Registros con salario USD: 5196
\end{verbatim}

Luego de todo el procesamiento ``de rigor'' que exige un análisis de
datos, nos queda un dataset limpio y listo para analizar. En este punto
comenzamos con la búsqueda de ``la combinación ganadora'', es decir, la
combinación de variables que maximiza el salario.

\begin{verbatim}
[INFO] Variables analizadas: 16
[INFO] Registros validos: 5196
\end{verbatim}

En esta parte realizamos un análisis por variable individual. Tratamos
de ir de a poco analizando la información que nos permita responder a la
pregunta central del trabajo.

\begin{verbatim}
[1] UBICACION GEOGRAFICA:
                  Media  Mediana      Q75         Max  Cantidad
es_caba                                                        
Resto del pais  5324.79  2124.29  3275.33  3885135.14      5196

GANADOR: Resto del pais
Mediana: $2,124 USD

[2] TIPO DE CONTRATO:
                                                       Media  Mediana  \
tipo_contrato                                                           
Contractor                                          11316.76  2641.51   
Staff (planta permanente)                            4477.06  2169.81   
Participación societaria en una cooperativa          2206.28  2075.47   
Tercerizado (trabajo a través de consultora o a...   1869.14  1600.73   
Freelance                                            1895.63  1391.51   

                                                        Q75         Max  \
tipo_contrato                                                             
Contractor                                          4379.25  3500000.00   
Staff (planta permanente)                           3154.64  3885135.14   
Participación societaria en una cooperativa         2650.60     4245.28   
Tercerizado (trabajo a través de consultora o a...  2264.15     9433.96   
Freelance                                           2405.66     9433.96   

                                                    Cantidad  
tipo_contrato                                                 
Contractor                                               859  
Staff (planta permanente)                               3768  
Participación societaria en una cooperativa               29  
Tercerizado (trabajo a través de consultora o a...       403  
Freelance                                                137  

GANADOR: Contractor
Mediana: $2,642 USD

[3] DEDICACION:
              Media  Mediana      Q75         Max  Cantidad
dedicacion                                                 
Full-Time   5507.10  2169.81  3301.89  3885135.14      4974
Part-Time   1240.13   849.06  1509.43     6603.77       222

GANADOR: Full-Time
Mediana: $2,170 USD

[4] MODALIDAD DE TRABAJO:
                                 Media  Mediana      Q75         Max  Cantidad
modalidad                                                                     
Híbrido (presencial y remoto)  4282.31  2207.55  3286.79  3500000.00      2056
100% remoto                    6650.02  2169.81  3396.23  3885135.14      2733
100% presencial                1692.13  1415.09  2125.50     6477.50       407

GANADOR: Híbrido (presencial y remoto)
Mediana: $2,208 USD

[5] SENIORITY:
               Media  Mediana      Q75         Max  Cantidad
seniority                                                   
Senior       7332.15  2830.19  4050.30  3885135.14      2648
Semi-Senior  3716.79  1886.79  2641.51  2600000.00      1634
Junior       2383.88  1224.81  1698.11   909118.18       914

GANADOR: Senior
Mediana: $2,830 USD

[6] PUESTO (Top 10):
                            Media  Mediana      Q75       Max  Cantidad
puesto                                                                 
Smart contracts engineer  7264.15  7264.15  7264.15   7264.15         1
embedded engineer         6750.00  6750.00  6750.00   6750.00         1
Engineer                  5566.04  5566.04  5566.04   5566.04         1
Staff Engineer            4961.24  4961.24  5813.95   6666.67         2
VP / C-Level              4715.65  4205.97  6376.50  10849.06        40
AI Engineer               4079.94  4096.19  4980.28   8867.92         7
Architect                 4117.63  3773.58  5188.68  12549.02       125
CIO                       3773.58  3773.58  3773.58   3773.58         1
GeneXus Analyst           3415.09  3415.09  3415.09   3415.09         1
Technical Leader          3580.64  3301.89  4525.02  11981.13       367
GANADOR: Smart contracts engineer
Mediana: $7,264 USD

[7] FORMA DE PAGO:
                                                       Media  Mediana  \
forma_pago                                                              
Cobro todo el salario en dólares                    11217.87  3301.89   
Cobro parte del salario en dólares                  13133.81  2340.42   
Mi sueldo está dolarizado (pero cobro en moneda...   5355.33  2075.47   

                                                        Q75         Max  \
forma_pago                                                                
Cobro todo el salario en dólares                    4866.50  3500000.00   
Cobro parte del salario en dólares                  3487.89  3885135.14   
Mi sueldo está dolarizado (pero cobro en moneda...  3292.92   909118.18   

                                                    Cantidad  
forma_pago                                                    
Cobro todo el salario en dólares                         811  
Cobro parte del salario en dólares                       714  
Mi sueldo está dolarizado (pero cobro en moneda...       323  

GANADOR: Cobro todo el salario en dólares
Mediana: $3,302 USD

[8] TAMANO DE EMPRESA:
                             Media  Mediana      Q75         Max  Cantidad
tamano_empresa                                                            
De 2001a 5000 personas     2912.02  2641.51  3720.12    11132.08       364
De 5001 a 10000 personas   2927.20  2547.17  3438.03    47142.86       247
Más de 10000 personas      2817.12  2547.17  3492.45    12000.00       589
De 1001 a 2000 personas    2874.95  2358.49  3584.91    10362.69       361
De 501 a 1000 personas    10830.66  2358.49  3590.61  3500000.00       432
De 201 a 500 personas      2633.27  2169.81  3301.89    12226.42       675
De 101 a 200 personas     12609.64  2169.81  3301.89  3500000.00       611
De 51 a 100 personas       8008.99  1886.79  2995.27  3885135.14       687
1 (solamente yo)           2685.63  1839.62  3820.75    10000.00        44
De 11  a 50  personas      3222.85  1720.10  2667.92   909118.18       880
De 2 a 10 personas         1863.54  1379.25  2172.89    10109.43       306

GANADOR: De 2001a 5000 personas
Mediana: $2,642 USD

[9] PERSONAS A CARGO:
                Media  Mediana      Q75         Max  Cantidad
tiene_equipo                                                 
Con equipo    8443.76  2830.19  4084.91  3500000.00      1498
Sin equipo    4061.35  1901.51  2869.27  3885135.14      3698

GANADOR: Con equipo
Mediana: $2,830 USD
\end{verbatim}

\subsubsection{El perfil del top 10\%}\label{el-perfil-del-top-10}

Si bien nuestro objetivo es encontrar la combinación ganadora, creemos
que un buen punto de partida es analizar el perfil del top 10\% con el
mejor salario. La idea es entender qué características tienen en común
los profesionales que están en este grupo y ver si podemos extraer
alguna conclusión útil para nuestro análisis.

\begin{verbatim}
*** Salario minimo Top 10%: $4,717 USD
*** Trabajadores en Top 10%: 530

[Distribucion Top 10%]

Ubicacion:
es_caba
Resto del pais    100.0
Name: proportion, dtype: float64

Tipo de Contrato:
tipo_contrato
Staff (planta permanente)                                 59.6
Contractor                                                35.1
Tercerizado (trabajo a través de consultora o agencia)     2.8
Freelance                                                  2.5
Name: proportion, dtype: float64

Seniority:
seniority
Senior         88.1
Semi-Senior    10.4
Junior          1.5
Name: proportion, dtype: float64

Top 5 Puestos:
puesto
Developer                  129
Manager / Director         116
Technical Leader            81
SysAdmin / DevOps / SRE     49
Architect                   39
Name: count, dtype: int64

Experiencia promedio: 14.3 anos
Edad promedio: 38.8 anos
Antiguedad promedio: 4.6 anos
\end{verbatim}

\subsubsection{El ``salto salarial'' en la trayectoria
profesional}\label{el-salto-salarial-en-la-trayectoria-profesional}

De a poco vemos emerger la infomación relevante. Una vez que analizamos
las variables anteriores, podemos abordar la cuestión de la
``trayectoria profesional'' para encontrar el momento del \textbf{salto
salarial}, \emph{¿suele tardar en llegar?}, \emph{¿llega en algún
momento o tiende más a bien a ser estable?}. Esto es lo que intentamos
responder a continuación.

\begin{verbatim}
EVOLUCIÓN SALARIAL POR EXPERIENCIA:

                        Cantidad  Mínimo      Q25  Mediana    Media      Q75  \
rango_exp                                                                      
0-2 años (Junior)            754  145.94   935.49  1273.16  2637.31  1756.84   
3-5 años (Semi-Senior)      1634    1.00  1389.47  1886.79  3716.79  2641.51   
6-8 años (Senior)            685    1.25  1840.20  2688.68  8085.04  3653.85   
9-12 años (Senior+)          608    1.37  1886.79  2754.73  3124.81  3920.56   
12+ años (Expert)           1355  145.99  1981.13  2978.77  8839.40  4339.62   

                            Máximo   Desv_Est  Crecimiento_%  Acumulado_%  
rango_exp                                                                  
0-2 años (Junior)        909118.18   33065.23            NaN     0.000000  
3-5 años (Semi-Senior)  2600000.00   64278.19      48.197399    48.197399  
6-8 años (Senior)       3500000.00  133633.65      42.500225   111.181627  
9-12 años (Senior+)       12679.25    1822.43       2.456596   116.369506  
12+ años (Expert)       3885135.14  141888.30       8.132920   133.966666  

MAYOR SALTO SALARIAL: 3-5 años (Semi-Senior) (+48.2%)
\end{verbatim}

\subsubsection{Evolución del salario a través de los
años}\label{evoluciuxf3n-del-salario-a-travuxe9s-de-los-auxf1os}

Podemos visualizar la evolución del salario mediano a lo largo de los
años de experiencia para observar tendencias y patrones en el
crecimiento salarial. También podemos analizar la distribución de
salarios en diferentes rangos de experiencia para ver cómo varía el
salario entre profesionales con diferentes niveles.

\pandocbounded{\includegraphics[keepaspectratio]{EPDS_1A_2C25_Trabajo_de_simulación_III_files/figure-pdf/cell-9-output-1.png}}

\pandocbounded{\includegraphics[keepaspectratio]{EPDS_1A_2C25_Trabajo_de_simulación_III_files/figure-pdf/cell-9-output-2.png}}

\subsubsection{Visualizaciones
comparativas}\label{visualizaciones-comparativas}

Ahora bien, una vez que tenemos toda esta información, podemos intentar
combinar las variables para encontrar la combinación ganadora que
maximiza el salario.

\pandocbounded{\includegraphics[keepaspectratio]{EPDS_1A_2C25_Trabajo_de_simulación_III_files/figure-pdf/cell-10-output-1.png}}

\begin{verbatim}

[TOP 10 COMBINACIONES]

Top 10 Combinaciones (Ubicación + Contrato + Dedicación):
                                                                                 Media  \
es_caba        tipo_contrato                                      dedicacion             
Resto del pais Contractor                                         Full-Time   12040.33   
               Staff (planta permanente)                          Full-Time    4563.36   
               Participación societaria en una cooperativa        Full-Time    2268.23   
               Freelance                                          Full-Time    2309.57   
               Tercerizado (trabajo a través de consultora o a... Full-Time    1912.36   
               Staff (planta permanente)                          Part-Time    1343.89   
               Contractor                                         Part-Time    1324.06   
               Tercerizado (trabajo a través de consultora o a... Part-Time     887.62   
               Freelance                                          Part-Time    1049.34   

                                                                              Mediana  \
es_caba        tipo_contrato                                      dedicacion            
Resto del pais Contractor                                         Full-Time   2830.19   
               Staff (planta permanente)                          Full-Time   2169.81   
               Participación societaria en una cooperativa        Full-Time   2124.69   
               Freelance                                          Full-Time   1836.32   
               Tercerizado (trabajo a través de consultora o a... Full-Time   1608.49   
               Staff (planta permanente)                          Part-Time   1008.90   
               Contractor                                         Part-Time    825.13   
               Tercerizado (trabajo a través de consultora o a... Part-Time    801.89   
               Freelance                                          Part-Time    745.28   

                                                                              Cantidad  
es_caba        tipo_contrato                                      dedicacion            
Resto del pais Contractor                                         Full-Time        801  
               Staff (planta permanente)                          Full-Time       3667  
               Participación societaria en una cooperativa        Full-Time         28  
               Freelance                                          Full-Time         92  
               Tercerizado (trabajo a través de consultora o a... Full-Time        386  
               Staff (planta permanente)                          Part-Time        101  
               Contractor                                         Part-Time         58  
               Tercerizado (trabajo a través de consultora o a... Part-Time         17  
               Freelance                                          Part-Time         45  
\end{verbatim}

\subsubsection{La combinación
ganadora}\label{la-combinaciuxf3n-ganadora}

Después de todo este análisis, podemos finalmente presentar la
combinación ganadora que maximiza el salario. Esta combinación se basa
en las variables que hemos analizado y las conclusiones que hemos
extraído a lo largo de este ``análisis exploratorio de datos'' (EDA).

\begin{verbatim}

PERFIL DE LOS TRABAJADORES MEJOR PAGOS

[1] UBICACION: Resto del pais
    Mediana: $2,124 USD

[2] TIPO CONTRATO: Contractor
    Mediana: $2,642 USD

[3] DEDICACION: Full-Time
    Mediana: $2,170 USD

[4] MODALIDAD: Híbrido (presencial y remoto)
    Mediana: $2,208 USD

[5] SENIORITY: Senior
    Mediana: $2,830 USD

[6] PUESTO: Smart contracts engineer
    Mediana: $7,264 USD

[7] FORMA PAGO: Cobro todo el salario en dólares
    Mediana: $3,302 USD

[8] TAMANO EMPRESA: De 2001a 5000 personas
    Mediana: $2,642 USD

[9] LIDERAZGO: Con equipo
    Mediana: $2,830 USD

EVOLUCION ESPERADA DEL SALARIO

0-2 años:  $1,273 USD
3-5 años:  $1,887 USD (+48.2%)
6-8 años:  $2,689 USD (+42.5%)
9-12 años: $2,755 USD (+2.5%)
12+ años:  $2,979 USD (+8.1%)

RECOMENDACIONES PARA MAXIMIZAR SALARIO

1. Ubicarse en Resto del pais
2. Buscar contratos tipo Contractor
3. Trabajar Full-Time
4. Priorizar modalidad Híbrido (presencial y remoto)
5. Desarrollarse hasta Senior
6. Especializarse en roles como Smart contracts engineer
7. Negociar modalidad "Cobro todo el salario en dólares"
8. Apuntar a empresas De 2001a 5000 personas
9. Desarrollar capacidad de liderazgo (Con equipo)
10. Acumular ~14 años de experiencia

Objetivo: Alcanzar Top 10% ($4,717+ USD/mes)
\end{verbatim}

\subsection{\texorpdfstring{Machine Learning: \emph{Random Forest
Regressor}}{Machine Learning: Random Forest Regressor}}\label{machine-learning-random-forest-regressor}

En esta parte del trabajo, buscamos ir más allá del análisis descriptivo
y explorar la importancia relativa de las variables en la predicción del
salario utilizando técnicas de Machine Learning. Nuestro objetivo es
identificar cuáles son las características más influyentes que
determinan el salario de los profesionales en el sector IT y ver si
existe una coincidencia entre estas variables y las que hemos
identificado en nuestro análisis previo como parte de la ``combinación
ganadora''. Creemos que al sumar este enfoque, podemos obtener una
visión más completa y robusta de los factores que impactan en el
salario, validando o complementando nuestras conclusiones anteriores con
un análisis basado en datos y modelos predictivos.

\subsubsection{Importancia de las
variables}\label{importancia-de-las-variables}

En primer lugar, debemos preparar los datos para el modelo de Machine
Learning. Esto incluye seleccionar las variables relevantes, manejar
valores faltantes y dividir los datos en conjuntos de entrenamiento y
prueba. El modelo que vamos a aplicar es un Random Forest Regressor, muy
utilizado para problemas de regresión que puede manejar tanto variables
numéricas como categóricas.

\begin{verbatim}
[INFO] Categorias de puestos creadas
puesto_categoria
Developer             1776
Management             843
Infra/Ops/Security     720
Data & Analytics       567
Other                  433
UX/UI Design           264
QA / Testing           247
Consultant             155
Architect              125
Product                 33
Agile / Scrum           33
Name: count, dtype: int64
\end{verbatim}

\pandocbounded{\includegraphics[keepaspectratio]{EPDS_1A_2C25_Trabajo_de_simulación_III_files/figure-pdf/cell-12-output-2.png}}

\begin{verbatim}
[INFO] Tamaño empresa simplificado:
tamano_empresa
Pequeña (1-100)       1917
Mediana (101-1000)    1718
Grande (1000+)        1561
Name: count, dtype: int64

[INFO] Registros iniciales: 1848

[LIMPIEZA DE DATOS]
Después de eliminar infinitos: 1848 registros
Después de eliminar salarios <= 0: 1848 registros
Rango válido: $259 - $12,198 USD
Después de filtrar outliers extremos: 1828 registros
Después de validar rangos de variables: 1827 registros

[INFO] Registros finales para modelo: 1827
[INFO] Variables predictoras: 11
[INFO] Salario USD - Min: $261 | Max: $12,106
[INFO] Salario USD - Media: $3,126 | Mediana: $2,642

[INFO] Features después de encoding: 26
[VALIDACIÓN] NaN en X: 0
[VALIDACIÓN] Infinitos en X: 0
[VALIDACIÓN] NaN en y: 0
[VALIDACIÓN] Infinitos en y: 0

[MODELO] Entrenamiento: 1461 | Test: 366
[MODELO] Entrenado exitosamente

MÉTRICAS DEL MODELO
MAE Train: $1,129 USD
El Error Absoluto Medio indica, en promedio, 
cuánto se alejan las predicciones del modelo de los datos reales.
MAE Test:  $1,097 USD
RMSE Test: $1,559 USD
El RMSE es la raíz cuadrada de la media de las diferencias 
al cuadrado entre los valores observados y los predichos. 
Siempre es no negativo, y los valores más bajos 
indican un modelo mejor ajustado.
R² Train:  0.4104 (41.04%)
R² explica la proporción de varianza de la variable dependiente que puede atribuirse 
a la variable (o variables) independiente(s). Podemos considerarlo como una medida de 
lo bien que nuestro modelo capta lo que los datos cuentan, 
y cuánto queda como ruido sin explicar.

R² Test:   0.3647 (36.47%)
Verificación del overfitting ("sobreajuste"):
Ocurre cuando el modelo se ajusta demasiado a sus datos de entrenamiento, 
impidiéndole realizar predicciones u obtener conclusiones precisas.

El modelo generaliza bien (diferencia R²: 0.046)

VALIDACIÓN CRUZADA (5-FOLD):
La validación cruzada K-Fold es una técnica que se utiliza para evaluar el rendimiento de 
los modelos de Machine Learning. Garantiza que el modelo generaliza bien.
R² promedio: 0.3445 (±0.0251)
R² por fold: ['0.371', '0.369', '0.340', '0.341', '0.301']
MAE promedio: $1,177 (±$56)
\end{verbatim}

\subsubsection{Visualización: predicciones
vs.~realidad}\label{visualizaciuxf3n-predicciones-vs.-realidad}

El modelo entrenado nos permite hacer predicciones sobre los salarios
basándonos en las características de los profesionales. Para evaluar el
rendimiento del modelo, podemos comparar las predicciones con los
valores reales de salario en el conjunto de prueba. Una forma efectiva
de visualizar esta comparación es mediante un gráfico de dispersión
(scatter plot), donde cada punto representa un profesional, con su
salario real en el eje Y y su salario predicho por el modelo en el eje
X.

\pandocbounded{\includegraphics[keepaspectratio]{EPDS_1A_2C25_Trabajo_de_simulación_III_files/figure-pdf/cell-13-output-1.png}}

\subsubsection{Cálculo de la importancia de las
variables}\label{cuxe1lculo-de-la-importancia-de-las-variables}

En este caso, el modelo nos proporciona una medida de la importancia de
cada variable en la predicción del salario. Podemos ordenar estas
importancias y visualizarlas para identificar cuáles son las
características más influyentes.

\begin{verbatim}
TOP 15 VARIABLES MÁS IMPORTANTES:

                                             Variable  Importancia_%
0                                         experiencia      25.301486
4                                    seniority_Senior      16.287362
23        forma_pago_Cobro todo el salario en dólares      11.255804
2                                                edad       9.608281
18                        puesto_categoria_Management       8.963168
12                     tamano_empresa_Pequeña (1-100)       4.362377
1                                          antiguedad       4.133615
3                               seniority_Semi-Senior       3.722887
25                            tiene_equipo_Sin equipo       2.774539
9             tipo_contrato_Staff (planta permanente)       2.697445
11                  tamano_empresa_Mediana (101-1000)       2.380986
24  forma_pago_Mi sueldo está dolarizado (pero cob...       1.727312
5                               modalidad_100% remoto       1.248300
13                         puesto_categoria_Architect       1.203275
22                      puesto_categoria_UX/UI Design       1.131085

IMPORTANCIA POR CATEGORÍA:

                Importancia_%
Categoria                    
Experiencia         25.301486
Seniority           20.010249
Puesto/Rol          13.288836
Forma de Pago       12.983116
Edad                 9.608281
Tamano Empresa       6.743363
Antiguedad           4.133615
Tipo Contrato        3.240205
Liderazgo            2.774539
Modalidad            1.916310
\end{verbatim}

\pandocbounded{\includegraphics[keepaspectratio]{EPDS_1A_2C25_Trabajo_de_simulación_III_files/figure-pdf/cell-14-output-2.png}}

\pandocbounded{\includegraphics[keepaspectratio]{EPDS_1A_2C25_Trabajo_de_simulación_III_files/figure-pdf/cell-14-output-3.png}}

\subsubsection{Análisis de residuos}\label{anuxe1lisis-de-residuos}

Finalmente, podemos analizar los residuos del modelo, es decir, la
diferencia entre los salarios reales y los predichos. Esto nos permite
identificar patrones en los errores del modelo y entender mejor su
rendimiento.

\begin{verbatim}
ESTADISTICAS DE LOS RESIDUOS:
Media: $33 USD
Mediana: $-249 USD
Desviación Estándar: $1,561 USD
\end{verbatim}

\pandocbounded{\includegraphics[keepaspectratio]{EPDS_1A_2C25_Trabajo_de_simulación_III_files/figure-pdf/cell-15-output-2.png}}

\subsubsection{Conclusiones del modelo de Machine
Learning}\label{conclusiones-del-modelo-de-machine-learning}

\begin{verbatim}

El modelo Random Forest revela que:

* Precision (R² test): 36.5% - Explica 36.5% de la variabilidad salarial. 
* Precision (R² CV): 34.5% - Validación cruzada confirma estabilidad
* Error promedio: $1,097 USD/mes (±$56). 
* Overfitting: Controlado (diferencia R²: 0.046)
* Factor mas determinante: Experiencia (25.3% de importancia)

Top 3 variables individuales:
1. experiencia: 25.30%
2. seniority_Senior: 16.29%
3. forma_pago_Cobro todo el salario en dólares: 11.26%
\end{verbatim}

El modelo identifica correctamente que la experiencia, edad y forma de
pago en dólares son los predictores más importantes del salario. Esto
coincide con nuestro análisis previo, donde también destacamos estas
variables como parte de la combinación ganadora. De esta forma validamos
nuestras conclusiones y reforzamos la idea de que estas características
son clave para maximizar el salario en el sector IT.

\subsection{Conclusiones}\label{conclusiones}

A modo de cierre, luego del EDA (``Análisis exploratorio de Datos'') y
del análisis utilizando Machine Learning, podemos llegar a la conclusión
de que la mejor combinación de variables para maximizar el salario en el
sector IT es la siguiente: en primer lugar, la experiencia laboral.
Podríamos decir que ``cuanto antes, mejor'', es decir, comenzar a
trabajar lo antes posible para acumular experiencia, incluso mientras se
está estudiando, ya que no parece ser un mercado laboral que exija
títulos universitarios para acceder a mejores salarios. La mejora
constante sí es algo necesario pero no necesariamente a través de
títulos formales. La dolarización del salario también es un factor
clave, ya que los salarios en dólares tienden a ser más altos en
comparación con los salarios en moneda local. Por último, algo que
también está vinculado con la seniority/experiencia es el liderazgo,
aquellas personas con ``gente a cargo'' que dirigen equipos suelen tener
salarios más altos. Esta última variable se vincula directamente con las
llamadas ``soft skills'', lo que coincide con la tendencia actual del
mercado laboral a valorar cada vez más estas habilidades interpersonales
y de gestión.




\end{document}
